\documentclass{beamer}
\usetheme{Dresden}

\title{Summary of Pattern Formation for Asynchronous Robots without Agreement in Chirality}
\author{Sruti Gan Chaudhuri, Swapnil Ghike, Shrainik Jain, Krishnendu Mukhopadhyaya}
\date{\today}

\begin{document}

\frame{\titlepage}

\begin{frame}{Introduction}
    This paper explores the collaborative tasks of autonomous mobile robots, focusing on a deterministic algorithm to arrange them into specific asymmetric patterns without explicit communication or agreement in coordinate systems.
\end{frame}

\begin{frame}{Methods}
    The study utilizes the CORDA model to design an algorithm that allows point robots to establish a common coordinate system and subsequently move to form the intended pattern by executing a series of deterministic movement strategies.
\end{frame}

\begin{frame}{Results}
    The proposed algorithm effectively demonstrates that robots can form any specified asymmetric pattern in finite time while ensuring collision-free movements and maintaining an invariant coordinate system throughout the process.
\end{frame}

\begin{frame}{Discussion}
    The findings indicate that even in the absence of agreement in chirality or coordinate systems, a distributed approach can lead to successful pattern formation, thereby enhancing the potential applications of swarms of autonomous robots in various environments.
\end{frame}

\begin{frame}{Conclusion}
    The algorithms developed not only achieve the formation of asymmetric patterns but can also be extended for use with fat robots, suggesting a promising avenue for future research in robotic coordination and pattern formation.
\end{frame}

\end{document}